\chapter{Verteiltes Rechnen}

Zusammenfassung der Vorlesung "`Verteiltes Rechnen"' aus dem Wintersemester 2017.\footnote{\url{https://www.scc.kit.edu/personen/11188.php}}

\section{Einführung}

\subsection{Verteilte Systeme und Middleware}
\begin{itemize}
	\item Definition: Zusammenschluss unabhängiger Computer zu einem einzelnen, kohärenten System
	\item Single-View; Unterschiede zwischen den verschiedenen Systemen werden vor dem Nutzer verborgen
	\item Anforderungen: Skalierbar, erweiterbar, fehlertolerant
	\item Beispiele: Workstations als \texttt{DesktopGrid}, WWW
	\item \textbf{Ziele}
	\begin{itemize}
		\item Verbinden von Nutzern und Ressourcen: Sicherer Zugang zu entfernten Ressourcen. Verteiltes System wird oft (aus ökonomischen Gründen) zwischen verschiedenen Institutionen geteilt \(\rightarrow\) AAA notwendig
		\item Transparenz: Abstrahieren von Datenzugriffen (bsp. Repräsentation, Byte-Order, Parallelität) und Lokation; Fehlerbehandlung
		\item Erweiterbar und portierbar durch standardisierte Protokolle/Schnittstellen/Semantiken
	\end{itemize}
\end{itemize}


\subsection{Web Services}
\begin{itemize}
	\item XML-basiert, plattformunabhängige Schnittstelle auf die mittels Webprotokoll zugegriffen werden kann \(\rightarrow\) ermöglicht lose Kopplung zwischen Systemen
	\item WSDL\footnote{Web Service Description Language} zur standardisierten Beschreibung. Stub/Skeleton-Klassen können automatisch generiert werden
\end{itemize}


\subsection{Web Services Resource Framework (WSRF)}
\begin{itemize}
	\item Generisches Framework zur Definition/Verwendung zustandsbehafteter Web Services (bsp. Ressourcenzugriff)
	\item Bestandteile: Ressourcen, Lifecycle-Management, Service-Gruppen und Fehlerbehandlung
\end{itemize}



\section{Grid}
\begin{itemize}
	\item Form des verteilten Rechnens, bei der ein virtueller Supercomputer aus einem Cluster lose gekoppelter Computer erzeugt wird\footnote{\ur{https://de.wikipedia.org/wiki/Grid-Computing}}
	\item Typische Eigenschaften: Loose gekoppelt, heterogen, geografische Verteilung, häufig Verwendung von Standardsoftware
	\item \textbf{Meta-Computing}
	\begin{itemize}
		\item Zusammenschluss unabhängiger, heterogener Supercomputer mittels Hochgeschwindkeits-WAN \(\rightarrow\) Verteilung einer einzelnen Anwendung über mehrere Supercomputer
		\item Beispiel: Gekoppelte Simulation, die ein einzelner Supercomputer nicht berechnen könnte
	\end{itemize}
	\item \textbf{Klassifikation}
	\begin{description}
		\item[Compute Grid:] Stellt Rechenkapazität zur Verfügung; weitere Spezifizierung in \textit{Desktop Grid} (oft Windows-basiert), \textit{Server Grids} (oft Unix-basiert) oder \textit{HPC-/Cluser-Grids}
		\item[Data Grids:] Stellt föderativen, sicheren, transparenten Hochgeschwindigkeitszugang zu Ressourcen zur Verfügung 
		\item[P2P Grids:] Geteilter Zugriff auf Speicher eines Desktoprechners, zentral oder dezentral
		\item[Collaborative Grids:] Stellen Kollaborationsdienste zur Verfügung. Bsp. Videokonferenzen, Chaträume, live-Messdaten
	\end{description}
\end{itemize}


\subsection{Architektur}
\begin{itemize}
	\item Verwendung von Middleware-Architekturen, die über Protokolle, APIs und SDK definiert sind
	\item Stundenglasmodell: Wenige zentrale Abstraktionen um möglichste viele Anwendungen zu unterstützen. Bsp. Internet: lediglich \texttt{IP} innerhalb des Netzwerk-Stacks vorgegeben
	\item \textbf{Schichtenarchitektur}
	\begin{description}
		\item[Collective Layer:] Verwaltung mehrerer Ressourcen mittels Verzeichnisdienste zum Allokieren, Schedulen, Verteilten/Monitoren/Replizieren
		\item[Resource Layer:] Ermöglicht teilen einzelner Ressourcen mittels Standardprotokolle zum Verbdingungsaufbau, Monitoring, Accounting, Bezahlen
		\item[Connectivity Layer:] Ermöglicht Verbindungen zwischen unterschiedlichen Fabric-Ressourcen (Transport, Routing, Naming, etc.), meist dem \texttt{IP}-Stack entnommen; Authentifizierung
		\item[Fabric Layer:] Physikalische Verbindung der geteilten Hardware-Ressourcen unter Berücksichtigung verschiedener Hardware-Typen und Verbindungstypen
	\end{description}
	\item \textbf{Struktur}
	\begin{itemize}
		\item Als standardisiertes Framework definiert: \textit{Open Grid Service Architecture} (OGSA)
		\item Service-orientiert \(\rightarrow\) alle Ressourcen werden durch einen Service repräsentiert; alle Komponenten sind virtuell
		\item Beinhaltet Basisschnittstellen für den Ressourcenzugriff mittels WSRF sowie Capabilities für Integration/Management der Services
		\item OGS Capabilities
		\begin{description}
			\item[Execution Management Services:] Finden und auswählen möglicher Ausführungsorte sowie Vorbereiten/Initiieren/Verwalten der Ausführung
			\item[Data Service:] Verwalten entfernter Daten, inklusive Staging/Replikation/Derivation und Metadaten
			\item[Resource Management Services:] Verwaltung der Ressourcen sowie der dazugehörigen \texttt{OSGA}-Infrastruktur
			\item[Security Services:] (Föderative-) Authentifikation und Autorisierung
			\item[Self-management Services:] Dient der automatische Grid-Verwaltung, inklusive Monitoring, Fehlertoleranz, Eigenreparatur, Analyse. Ziel: Anpassen der Konfiguration um SLA\footnote{Service-Level-Agreement} einzuhalten
			\item[Information Services:] Naming, Service-/Ressourcendiscovery, Logging/Monitoring
		\end{description}
	\end{itemize}
\end{itemize}


\subsection{Sicherheit}
\begin{itemize}
	\item Grid-Ressourcen kommunizieren über das Internet \(\rightarrow\) Angreifer können Nachrichten lesen/ändern/löschen/hinzufügen/etc.
	\item Voraussetzungen: 
\end{itemize}

















