\chapter{Verteiltes Rechnen}

Zusammenfassung der Vorlesung "`Verteiltes Rechnen"' aus dem Wintersemester 2017.\footnote{\url{https://www.scc.kit.edu/personen/11188.php}}

\section{Einführung}

\subsection{Verteilte Systeme und Middleware}
\begin{itemize}
	\item Definition: Zusammenschluss unabhängiger Computer zu einem einzelnen, kohärenten System
	\item Single-View; Unterschiede zwischen den verschiedenen Systemen werden vor dem Nutzer verborgen
	\item Anforderungen: Skalierbar, erweiterbar, fehlertolerant
	\item Beispiele: Workstations als \texttt{DesktopGrid}, WWW
	\item \textbf{Ziele}
	\begin{itemize}
		\item Verbinden von Nutzern und Ressourcen: Sicherer Zugang zu entfernten Ressourcen. Verteiltes System wird oft (aus ökonomischen Gründen) zwischen verschiedenen Institutionen geteilt \(\rightarrow\) AAA notwendig
		\item Transparenz: Abstrahieren von Datenzugriffen (bsp. Repräsentation, Byte-Order, Parallelität) und Lokation; Fehlerbehandlung
		\item Erweiterbar und portierbar durch standardisierte Protokolle/Schnittstellen/Semantiken
	\end{itemize}
\end{itemize}


\subsection{Web Services}
\begin{itemize}
	\item XML-basiert, plattformunabhängige Schnittstelle auf die mittels Webprotokoll zugegriffen werden kann \(\rightarrow\) ermöglicht lose Kopplung zwischen Systemen
	\item WSDL\footnote{Web Service Description Language} zur standardisierten Beschreibung. Stub/Skeleton-Klassen können automatisch generiert werden
\end{itemize}


\subsection{Web Services Resource Framework (WSRF)}
\begin{itemize}
	\item Generisches Framework zur Definition/Verwendung zustandsbehafteter Web Services (bsp. Ressourcenzugriff)
	\item Bestandteile: Ressourcen, Lifecycle-Management, Service-Gruppen und Fehlerbehandlung
\end{itemize}



\section{Grid}
\begin{itemize}
	\item Form des verteilten Rechnens, bei der ein virtueller Supercomputer aus einem Cluster lose gekoppelter Computer erzeugt wird\footnote{\url{https://de.wikipedia.org/wiki/Grid-Computing}}
	\item Typische Eigenschaften: Loose gekoppelt, heterogen, geografische Verteilung, häufig Verwendung von Standardsoftware
	\item \textbf{Checkliste nach Foster}
	\begin{itemize}
		\item Keine zentrale Kontrolle
		\item Verwendung allgemeiner, standardisierter, quelloffener Software und Schnittstellen
		\item Sehr hohe QoS-Anforderungen (Antwortzeit, Durchsatz, Sicherheit, Verfügbarkeit, etc.). Mehrwert des Systems insgesamt höher als die Summe der Einzelsysteme 
	\end{itemize}
	\item \textbf{Meta-Computing}
	\begin{itemize}
		\item Zusammenschluss unabhängiger, heterogener Supercomputer mittels Hochgeschwindkeits-WAN \(\rightarrow\) Verteilung einer einzelnen Anwendung über mehrere Supercomputer
		\item Beispiel: Gekoppelte Simulation, die ein einzelner Supercomputer nicht berechnen könnte
	\end{itemize}
	\item \textbf{Klassifikation}
	\begin{description}
		\item[Compute Grid:] Stellt Rechenkapazität zur Verfügung; weitere Spezifizierung in \textit{Desktop Grid} (oft Windows-basiert), \textit{Server Grids} (oft Unix-basiert) oder \textit{HPC-/Cluser-Grids}
		\item[Data Grids:] Stellt föderativen, sicheren, transparenten Hochgeschwindigkeitszugang zu Ressourcen zur Verfügung 
		\item[P2P Grids:] Geteilter Zugriff auf Speicher eines Desktoprechners, zentral oder dezentral
		\item[Collaborative Grids:] Stellen Kollaborationsdienste zur Verfügung. Bsp. Videokonferenzen, Chaträume, live-Messdaten
	\end{description}
\end{itemize}


\subsection{Architektur}
\begin{itemize}
	\item Verwendung von Middleware-Architekturen, die über Protokolle, APIs und SDK definiert sind
	\item Stundenglasmodell: Wenige zentrale Abstraktionen um möglichste viele Anwendungen zu unterstützen. Bsp. Internet: lediglich \texttt{IP} innerhalb des Netzwerk-Stacks vorgegeben
	\item \textbf{Anatomie: Schichtenarchitektur}
	\begin{itemize}
		\item Schichten
		\begin{description}
			\item[Collective Layer:] Verwaltung mehrerer Ressourcen mittels Verzeichnisdienste zum Allokieren, Schedulen, Verteilten/Monitoren/Replizieren
			\item[Resource Layer:] Ermöglicht Teilen einzelner Ressourcen mittels Standardprotokolle zum Verbindungsaufbau, Monitoring, Accounting, Bezahlen. Implementierung nutzt \textit{Fabric-Layer}-Methoden
			\item[Connectivity Layer:] Ermöglicht Verbindungen zwischen unterschiedlichen Fabric-Ressourcen (Transport, Routing, Naming, etc.), meist dem \texttt{IP}-Stack entnommen; Authentifizierung
			\item[Fabric Layer:] Physikalische Verbindung der geteilten Hardware-Ressourcen unter Berücksichtigung verschiedener Hardware-Typen und Verbindungstypen
		\end{description}
		\item Resource-Layer vs. Fabric-Layer: \textit{Fabric-Layer} umfasst die Plattform (Hardware und Betriebssystem); \textit{Resource-Layer} umfasst virtuelle, abstrakte Ressourcen
	\end{itemize}
	\item \textbf{Physiologie}
	\begin{itemize}
		\item Implementierung mittels OGSA\footnote{Open Grid Service Architecture}: Service-orientiertes Modell, auf dessen Ressourcen per WSRF\footnote{Web Services Resource Framework} zugegriffen wird
		\item Service-orientiert \(\rightarrow\) alle Ressourcen werden durch einen Service repräsentiert; alle Komponenten sind virtuell
		\item Beinhaltet Basisschnittstellen für den Ressourcenzugriff mittels WSRF sowie Capabilities für Integration/Management der Services
		\item OGS Capabilities
		\begin{description}
			\item[Execution Management Services:] Finden und auswählen möglicher Ausführungsorte sowie Vorbereiten/Initiieren/Verwalten der Ausführung
			\item[Data Service:] Verwalten entfernter Daten, inklusive Staging/Replikation/Derivation und Metadaten
			\item[Resource Management Services:] Verwaltung der Ressourcen sowie der dazugehörigen \texttt{OSGA}-Infrastruktur
			\item[Security Services:] (Föderative-) Authentifikation und Autorisierung
			\item[Self-management Services:] Dient der automatische Grid-Verwaltung, inklusive Monitoring, Fehlertoleranz, Eigenreparatur, Analyse. Ziel: Anpassen der Konfiguration um SLA\footnote{Service-Level-Agreement} einzuhalten
			\item[Information Services:] Naming, Service-/Ressourcendiscovery, Logging/Monitoring
		\end{description}
	\end{itemize}
	\item Anatomie vs. Physiologie: \textit{Anatomie} beschreibt die Architektur eines Grids als Middleware; \textit{Physiologie} beschreibt die Kommunikation mittels OSGA
\end{itemize}


\subsection{Sicherheit}
\begin{itemize}
	\item Grid-Ressourcen kommunizieren über das Internet \(\rightarrow\) Angreifer können Nachrichten lesen/ändern/löschen/hinzufügen/etc.
	\item Voraussetzungen: % TODO
\end{itemize}


\subsection{Infrastruktur}
\begin{itemize}
	\item \textbf{Worldwide LHC Computing Grid (WLCG)} % TODO
	\begin{itemize}
		\item 
	\end{itemize}
\end{itemize}


\subsection{Job Submission}
\begin{enumerate}
	\item Resource Discovery: Statisches Finden der passenden Ressourcen, die für die Job-Ausführung benötigt werden (und zu denen der Benutzer Zugang hat). Problem: Während der Discovery sind ggf. noch nicht alle Details der Ressourcen bekannt
	\item Resource Selection and Allocation:
	\item Ausführung und Job-Verwaltung:
\end{enumerate}



\section{Big Data}

\subsection{Einführung}
\begin{itemize}
	\item \textbf{Charakteristika\footnote{\url{https://en.wikipedia.org/wiki/Big_data\#Characteristics}}}
	\begin{description}
		\item[Volume:] Die Menge der erzeugten/generierten Daten \(\rightarrow\) gibt Aufschluss über den Wert der Daten und ob es sich um "`Big Data"' handelt
		\item[Velocity:] Geschwindigkeit, in der die Daten gewonnen/verarbeitet werden. Bei Big Data oft in Echtzeit
		\item[Variety:] Typ/Art der Daten. Ermöglicht die effektive Nutzung der Daten. Big Data wird aus verschiedensten Quellen gewonnen und verknüpft
		\item[Value:] (Nicht-)Reproduzierbarkeit der Daten
		\item[Variability:] Inkonsistenzen der Daten können Verwaltungsprozesse verhindern oder erschweren
		\item[Veracity:] Qualität der Daten (bspw. Korrektheit) \(\rightarrow\) maßgeblich für Ausgabequalität
	\end{description}
	\item \textbf{Anwendung Charakteristika auf HLC\footnote{Large Hadron Collider}}
	\begin{description}
		\item[Volume:] Sehr wichtig, da mehrere Terabyte pro Sekunde erzeugt werden
		\item[Velocity:] Nicht so wichtig, da die Aggregation nicht während des Experiments erfolgen muss
		\item[Variety:] Unwichtig, nur sehr wenige Datentypen
		\item[Value:] Unwichtig, da das Experiment wiederholt werden kann
		\item[Variability:] Unwichtig, da sich die Datentypen nicht ändern 
	\end{description}
\end{itemize}


\subsection{Data Stewardship, Curation and Preservation}
\begin{itemize}
	\item \textbf{Data Stewardship}
	\begin{itemize}
		\item Für die Nutzung und die Sicherheit digitaler Güter verantwortlich. Kümmert sich um Pflege, Verwaltung und Zugang
		\item Stellt \textit{Data Preservation} sicher und unterstützt \textit{Data Curation}
	\end{itemize}
	\item \textbf{Data Preservation}
	\begin{itemize}
		\item Ziel: Sicherstellung der Langzeit-Brauchbarkeit und Verfügbarkeit digitaler Güter
		\item Beeinhaltet Authentizität (digitales Logbuch), Verfügbarkeit (trusted Source), Nutzbarkeit (Formate, VMs, Emulation,...) und Integrität
		\item Speicherformate etc. müssen regelmäßig aktualisiert werden
	\end{itemize}
	\item \textbf{Data Curation}
	\begin{itemize}
		\item Verwaltung und Wertsteigerung digitaler Güter \(\rightarrow\) Verwaltung der Metadaten
		\item Beinhaltet Zugang (inklusive Authentifizierung) und Rechteverwaltung
		\item Tätigkeiten: Hinzufügen von Erläuterungen; Erstellen von Verknüpfungen; Verbessern und Aktualisieren der Dokumentation
	\end{itemize}
\end{itemize}


\subsection{MapReduce}
\begin{itemize}
	\item Skalierbares, massiv-paralleles Programmiermodell zum Verarbeiten extrem großer Datenmengen in Form von Key-Value-Paaren
	\item Ungeeignet bei Ahängigkeiten zwischen den Daten oder serieller Ausführung \(\rightarrow\)  nicht parallelisierbar
	\item \textbf{Vorgehen}
	\begin{description}
		\item[Map:] Durchführen einer Map-Operation pro KV-Paar. Erzeugt \textit{Intermediate Values}
		\item[Reduce:] Zusammenfassen der selben \textit{Intermediate Values}. Eine \texttt{Reduce}-Instanz pro \textit{Intermediate}-Key mit einem Iterator auf sämtliche Values zu diesem Key
	\end{description}
	\item Schritte: Split, Map, Shuffle, Reduce
	\item \textbf{Beispiele}
	\begin{itemize}
		\item \texttt{WordCount}
		\begin{description}
			\item[Eingabe:] Liste mit Dokumenten; jede \texttt{Map}-Instanz gibt pro Wort ein KV-Paar weiter
			\item[Intermediate Value:] \texttt{EmitIntermediate(w, 1);}
			\item[Ausgabe:] Jede \texttt{Reduce}-Operation ermittelt über den Iterator die Häufigkeit des jeweiligen Wortes
		\end{description}
		\item \texttt{PageRank}
		\begin{itemize}
			\item Formel
			\begin{itemize}
				\item \(PR(u) = \frac{1-\beta}{N} + \beta \sum_{v \in B_u}\frac{PR(v)}{L(v)}\)
				\item \(\beta\): Dämpfungsfaktor, meist \(\beta \approx 0.8\). Verhindert Dead-Ends und Selbstverlinkung ("`Spider-Trap"')
				\item \(B_u\): Liste aller Seiten, die auf Seite \(u\) verweisen
				\item \(L_v\): Anzahl der ausgehenden Links von Seite \(v\)
			\end{itemize}
			\item Implementierung
			\begin{description}
				\item[Map:] Weitergabe von KV-Paaren der Form \texttt{\(\big(\)Ziel ausgehender Link, Anteil Quell-PageRank geteilt durch Anzahl ausgehender Links\(\big)\)}
				\item[Reduce:] Summe über die Elemente des Iterators
			\end{description}
		\end{itemize}
	\end{itemize}
\end{itemize}



\section{Cloud}

\subsection{Einführung}
\begin{itemize}
	\item \textbf{Abgrenzung gegenüber Grid-Computing}
	\begin{itemize}
		\item Checkliste
		\begin{itemize}
			\item Zentrale Kontrolle
			\item Verwendung proprietärer Protokolle und Schnittstellen
			\item Erfüllen eher "`einfache"' Aufgaben
		\end{itemize}
		\item Weitere Unterschiede
		\begin{itemize}
			\item Einsatz: Cloud kommerziell (Peakleistungsverteilung zwischen mehreren Kunden), Grid in der Forschung
			\item Grid-Rechner global verteilt, Cloud in Rechenzentren
		\end{itemize}
	\end{itemize}
	\item \textbf{Charakteristik (NIST)}
	\begin{description}
		\item[On-Demand Self-Service:] Kunden können Ressourcen automatisch, ohne Freischaltung durch Technikerpersonal, hinzubuchen
		\item[Broad Netwerk Access:] Ressourcen sind mittels Standardprotokolle über das Netzwerk zugänglich
		\item[Resource Pooling:] Die Ressourcen des Providers werden zusammengefasst mehreren Kunden zur Verfügung gestellt
		\item[Rapid Elastizität:] Kunden können Ressourcen dynmaisch verringern und erweitern (scale-in, scale-out)
		\item[Measured Service:] Die Ressourcenverwendung durch den Kunden wird automatisch verwaltet und optimiert. Abrechnungsgrundlage des Providers
	\end{description}
	\item \textbf{Scale-Out vs. Scale-Up}
	\begin{description}
		\item[Scale-Out:] Hinzufügen weiterer Ressourcen-Einheiten (beispielsweise VMs) \(\rightarrow\) horizontale Skalierung
		\item[Scale-Up:] Verbesserung der einzelnen Ressourcen-Einheiten (beispielsweise bessere CPUs in VMs) \(\rightarrow\) vertikale Skalierung
	\end{description}
	\item \textbf{Service-Modelle}
	\begin{description}
		\item[IaaS:] Infrastructure as a Service
		\item[PaaS:] Platform as a Service
		\item[SaaS:] Software as a Service
	\end{description}
\end{itemize}


\subsection{Virtualisierung}
\begin{itemize}
	\item \textbf{Vorteile}
	\begin{itemize}
		\item Betriebssystem und Ressourcen können on-demand angepasst werden \(\rightarrow\) einfache Verwaltung
		\item Gäste sind isoliert (Sicherheit, dedizierte Leistung, etc.)
		\item Gäste können unkompliziert migriert werden
		\item Verwendung von Legacy-Systemen
	\end{itemize}
	\item \textbf{Virtualisierung vs. Emulation}
	\begin{description}
		\item[Virtualisierung:] Instruktionen des Gastsystemes werden (so weit wie möglich) direkt durch die Hardware des Hostsystems ausgeführt
		\item[Emuation:] Instruktionen des Gastsystems werden vom Host nachgebildet und ausgeführt. Ermöglicht die Ausführung von Fremdarchitekturen
	\end{description}
	\item Kritische Instruktionen: Sensitive, aber nicht privilegierte Instruktionen. 17 bei x86, beispielsweise \text{POPF} \(\rightarrow\) trappen nicht\footnote{\url{https://de.wikipedia.org/wiki/Virtualisierungsforderungen_von_Popek_und_Goldberg}}
	\item \textbf{Virtualisierungstypen}
	\begin{itemize}
		\item Vollvirtualisierung: \textit{Binary Translation} zur Laufzeit zum Finden und Ersetzen kritischer Anweisungen; User-Level-Anweisungen werden direkt ausgeführt (native Geschwindigkeit). Gast "`weiß"' nicht, dass er in einer virtuellen Umgebung läuft (\texttt{Ring 1})
		\item Paravirtualisierung: Kritische Instruktionen werden vor der Ausführung durch \textit{Hypercalls} ersetzt und kommunizieren direkt mit dem Hypervisor. Gast "`weiß"', dass er virtuell läuft (\texttt{Ring 1})
		\item Hardware-assistierte Virtualisierung (VT-x, AMD-V): Zusätzlicher Ausführungsmodus für Gastsysteme
		\item Speicherirtualisierung: Der physische Speicher des Gastes wird in den physischesn Speicher des Hypervisors gemappt \(\rightarrow\) Gast hat keinen direkten Zugriff aus den Arbeitsspeicher des Hosts
	\end{itemize}
	\item \textbf{Docker}
	\begin{itemize}
		\item Container: Leichtgewichtige, vollständige Ausführungsumgebung für Anwendungen. Beinhaltet Libraries und Konfigurationen
		\item Verwendet den Betriebssystem-Kernel \(\rightarrow\) meist sehr gute Performance bei wenig Overhead
	\end{itemize}
\end{itemize}
