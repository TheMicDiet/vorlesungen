\chapter{Verteiltes Rechnen}

Zusammenfassung der Vorlesung "`Verteiltes Rechnen"' aus dem Wintersemester 2017.\footnote{\url{https://www.scc.kit.edu/personen/11188.php}}

\section{Einführung}

\subsection{Verteilte Systeme und Middleware}
\begin{itemize}
	\item Definition: Zusammenschluss unabhängiger Computer zu einem einzelnen, kohärenten System
	\item Single-View; Unterschiede zwischen den verschiedenen Systemen werden vor dem Nutzer verborgen
	\item Anforderungen: Skalierbar, erweiterbar, fehlertolerant
	\item Beispiele: Workstations als \texttt{DesktopGrid}, WWW
	\item \textbf{Ziele}
	\begin{itemize}
		\item Verbinden von Nutzern und Ressourcen: Sicherer Zugang zu entfernten Ressourcen. Verteiltes System wird oft (aus ökonomischen Gründen) zwischen verschiedenen Institutionen geteilt \(\rightarrow\) AAA notwendig
		\item Transparenz: Abstrahieren von Datenzugriffen (bsp. Repräsentation, Byte-Order, Parallelität) und Lokation; Fehlerbehandlung
		\item Erweiterbar und portierbar durch standardisierte Protokolle/Schnittstellen/Semantiken
	\end{itemize}
\end{itemize}


\subsection{Web Services}
\begin{itemize}
	\item XML-basiert, plattformunabhängige Schnittstelle auf die mittels Webprotokoll zugegriffen werden kann \(\rightarrow\) ermöglicht lose Kopplung zwischen Systemen
	\item WSDL\footnote{Web Service Description Language} zur standardisierten Beschreibung. Stub/Skeleton-Klassen können automatisch generiert werden
\end{itemize}


\subsection{Web Services Resource Framework (WSRF)}
\begin{itemize}
	\item Generisches Framework zur Definition/Verwendung zustandsbehafteter Web Services (bsp. Ressourcenzugriff)
	\item Bestandteile: Ressourcen, Lifecycle-Management, Service-Gruppen und Fehlerbehandlung
\end{itemize}