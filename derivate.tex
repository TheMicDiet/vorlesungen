\chapter{BWL: Derivate}

Zusammenfassung der Vorlesung "`Derivate"' aus dem Sommersemester 2016.\footnote{\url{https://derivate.fbv.kit.edu/943.php}}

\section{Teil I - Einführung}

\subsection{Begriffliche Grundlagen}
\begin{itemize}
	\item Derivate Finanzinstrumente: Finanzkontrakte, deren Wert durch andere, grundlegendere Größen (Basiswert, Underlying) bestimmt wird
	\item Beispiele: Optionen (aktiver Börsenhandel), Forwards oder Swaps (over-the-Counter)
	\item Basiswerte: Grundsätzliche jede beliebige Größe möglich, häufig jedoch Preise börsengehandelter Wertpapiere, bzw. davon abgeleitete Größen. Beispiele: Aktien, Aktienindizes, Anleihen, Zinssätze, Wechselkurse, andere Derivate
	\item \textbf{Kassageschäft versus Termingeschäft}
	\begin{itemize}
		\item Kassageschäft: Sofortiger Kauf der Aktie A zum aktuellen Kassapreis. Vertragsabschluss und -erfüllung zugleich
		\item Termingeschäft: Vereinbarung, die Aktie A in einem zukünftigen Zeitpunkt zu kaufen. Vertragsabschluss und -erfüllung zu unterschiedlichen Zeitpunkten
		\begin{itemize}
			\item Forward auf Aktie A: Erfüllung verpflichtend (\textit{unbedingtes Termingeschäft})
			\item Option auf Aktie A: Erfüllung für den Optionskäufer nicht verpflichtend (\textit{bedingtes Termingeschäft})
		\end{itemize}
	\end{itemize}
	\item \textbf{Typische Handelsmotive}
	\begin{itemize}
		\item Hedging: Reduktion des Risikos einer bestehenden der zukünftig aufzubauenden Kassaposition
		\item Spekulation: Aufbau einer Position zur Umsetzung von Erwartungen
		\item Arbitrage: Ausnutzung von Preisunterschieden auf verschiedenen Märkten
	\end{itemize}
\end{itemize}

\subsubsection{Optionskontrakte}
Inhaber (Verkäufer, long) besitzt das Recht, aber nicht die Pflicht, vom Vertragspartner (Verkäufer, Stillhalter, short) in einem oder mehreren zukünftigen Zeitpunkten die Erfüllung der eingegangen Verpflichtung zu verlangen. Beispielsweise bei Finanztiteln die Lieferung zu einem festgelegten Preis.

\subsubsection{Forward- und Future-Kontrakte}
\begin{itemize}
	\item Forward: Vereinbarung zweier Vertragspartner, den Kontraktgegenstand (Basiswert) in einem zukünftigen Zeitpunkt zu einem festgelegten Preis (Forwardpreis) zu kaufen oder zu verkaufen
	\item Future: Vereinbarung analog zu Forward, i.d.R. börsengehandelter Kontrakt, daher standardisiert. Wesentliche Unterschiede: Täglicher Gewinn- und Verlustausgleich (marking to market) sowie häufig Lieferoption für den Verkäufer
\end{itemize}


\subsection{Grundidee der Derivatebewertung}
Annahme: Wertpapierpreise stellen sich so ein, dass keine risikolosen Gewinne (Arbitrage) möglich sind. Bei dauerhafter Verletzung dieser Forderung würde jeder nicht-gesättigte Investor Arbitragmöglichkeiten ausnutzen und beliebig reich werden können.

\subsubsection{No-Arbitrage-Definitionen}
\begin{itemize}
	\item \textbf{Typ 1:} Es gibt kein geschenktes Lotterielos mit positiver Gewinnchance
	\begin{itemize}
		\item Kein Kapitaleinsatz in \(t=0\)
		\item Wert der Position \(\ge 0\) mit Wahrscheinlichkeit \(1\)
		\item Wert der Position \(> 0\) mit positiver Wahrscheinlichkeit
	\end{itemize}
	\item \textbf{Typ 2:} Ex gibt kein Geschenk ohne zukünftige Verpflichtung - \textit{No-Free-Lunch}
	\begin{itemize}
		\item Mittelzufluss in \(t=0\)
		\item Für einen zukünftigen Zeitpunkt \(t>0\) gilt: Wert der Position \(\ge 0\) mit Wahrscheinlichkeit \(1\)
	\end{itemize}
	\item Gesetz des einen Preises: Identische zukünftige Zahlungsströme bedeuten identische Werte heute
\end{itemize}

\subsubsection{Arbitrage- und gleichgewichtsorientierte Bewertung}
\begin{itemize}
	\item \textbf{Gleichgewichtsorientiert:} Explizite Modellierung des individuellen Risikos und Nutzenkalküls sowie Markträumung \(\rightarrow\) abhängig von beobachtbaren Größen wie Risikopräferenzen, Aussattungen, Planugshorizonten etc.
	\item \textbf{Arbitrageorientiert:} Replikation der zukünftigen Zahlungen durch Basiswertpapiere sowie Arbitragefreiheit \(\rightarrow\) Ergebnisse bei gegebener Dynamik der Preise der Basiswertpapiere präferenzfrei. Spezialfall des allgemeinen gleichgewichtsorientierten Ansatzes und funktioniert daher nicht immer (s.u.)
	\begin{itemize}
		\item Reine Relativbewertung, führt zu praktikablen Bewertungsmodellen, deren Eingangsgrößen weitgehend objektiv ermittelbar sind
		\item Funktioniert nicht, wenn Werte beispielsweise im wesentlichen von nichthandelbaren Absatzrisikien abhängen (Bsp.: Wetterderivat)
	\end{itemize}
\end{itemize}



\section{Teil II - Forwards und Futures}

\subsection{Arbitragefreie Terminpreise}
\begin{itemize}
	\item Annahmen: Keine Transaktionskosten oder Steuern oder Informationskosten oder Leerverkaufsbeschränkungen, beliebige Teilbarkeit der Wertpapiere, risikolose Mittelanlage, lagerfähiger Basiswert (perfekter Markt)
	\item \textbf{Ermittlung des fairen (arbitragefreien) Terminpreises \(f(t,T))\): Cash und Carry Arbitrage}
	\begin{itemize}
		\item Forward löst bei Abschluss keine Zahlung aus und besitzt einen Wert von \(0\), d.h. der Terminpreis \(f(t,T)\) ist gerade so festgelegt, dass das Geschäft für beide Parteien fair ist
		\item Prinzip: Synthetisches Erzeugen des Forwards durch soforten Kauf des lagerfähigen Basiswerts sowie Kreditaufnahme in Höhe des aktuellen Preises
		\item Im einfachsten Fall entspricht der faire Terminpreis dem aufgezinsten Kassapreis (ohne Haltekosten etc.)
		\item Basis konvergiert am Ende immer gegen \(0\), da zu diesem Zeitpunkt das Termingeschäft einem Kassageschäft entspricht
	\end{itemize}
	\item \textbf{Beispiele für Terminpreise}
	\begin{itemize}
		\item Ertragloser Basiswert: \(f(0,T) = S(0)\cdot(1+r)^t \approx S(0) \cdot e^{rT}\)
		\item Ertragsloser Basiswert mit erwarteter Rendite \(q\): \(f(t,T) = S(t) \cdot e^{(r-q)\cdot (T-t)}\)
		\item Einmalige, sichere Zahlung \(X\) in \(t_1\) (Bsp.: Dividenden- oder Kuponzahlung). Alle Erträge werden entsprechend Diskontiert: \(f(0,T) = \big(S(0)-X(0)\big)\cdot (1+r)^T = \big(S(0)-X(0)\big)\cdot e^{rT}\)
	\end{itemize}
\end{itemize}



\section{Teil III - Optionen}

\subsection{Grundlagen}
\begin{itemize}
	\item Standardoptionen: Kaufoptionen (Calls) und Verkaufsoptionen (Puts)\footnote{\url{https://de.wikipedia.org/wiki/Option_(Wirtschaft)\#.C3.9Cbersicht}}
	\item Der Käufer hat das Recht – nicht jedoch die Pflicht – zu bestimmten Ausübungszeitpunkten eine bestimmte Menge des Bezugswerts zu einem zuvor festgelegten Ausübungspreis (oder englisch strike) zu kaufen oder zu verkaufen
	\item Der Verkäufer der Option (auch Stillhalter, Schreiber, Zeichner) erhält den Kaufpreis der Option. Er ist im Falle der Ausübung verpflichtet, den Basiswert zum vorher bestimmten Preis zu verkaufen bzw. zu kaufen
	\item Klassische Analyse von Optionspositionen anhand von Gewinn- und Verlust-Diagrammen. Ableitbarkeit des maximaler Gewinn/Verlust in \(T\)
	\item \textbf{Ausübungsarten}
	\begin{itemize}
		\item Europäische Option: Die Option kann nur am Fälligkeitsdatum ausgeübt werden
		\item Amerikanische Option: Die Option kann an jedem Handelstag vor der Fälligkeit ausgeübt werden
	\end{itemize}
	\item \textbf{Arten von Optionen}
	\begin{itemize}
		\item Call long
		\begin{itemize}
			\item Motivation: Käufer erwartet einen Preisanstieg bei einem Finanzprodukt und hofft dieses günstig er erwerben
			\item Ausübungswert: \(C(T) = max(S(T)-X,0)\)
			\item Gewinn/Verlust: \(GV(T)=C(T)-C(0) = max(S(T)-X,0)-C(0)\)
			\item Gewinn/Verlust, inklusive Zinsen: \(GV(T)=C(T)-C(0) = max(S(T)-X,0)-C(0) \cdot (1+r)^T\)
		\end{itemize}
		\item Put long
		\begin{itemize}
			\item Motivation: Käufer erwartet einen Preisverfall bei einem Finanzprodukt und hofft dieses teuer verkaufen zu können
			\item Ausübungswert: \(P(T)=max(X-S(T),0)\)
			\item Gewinn/Verlust: \(GV(T)=P(T)-P(0) = max(X-S(T),0)-P(0)\)
			\item Gewinn/Verlust, inklusive Zinsen: \(GV(T)=P(T)-P(0) = max(X-S(T),0)-P(0) \cdot (1+r)^T\)
		\end{itemize}
	\end{itemize}
	\item \textbf{Zusammengesetzte Positionen}
	\begin{itemize}
		\item Hedge
		\begin{itemize}
			\item Kombination von Wertpapieren mit Optionen
			\item 1:1 Hedge: Gleiche Anzahl von Wertpapieren und Optionen, Ratio Hedge: Jeweils flexible, ungleiche Anzahl
			\item Covered Call (1:1 Hedge): Bei einem Covered Call erwirbt man einen Basiswert und verkauft gleichzeitig eine Call-Option auf diesen. Dadurch ist die offene Position im Call durch den Basiswert "`gedeckt"'. Der Ertrag der Strategie stammt allein aus dem Verkauf der Option. Dafür darf der Basiswert zum Ausübungsdatum nicht unter den Ausübungspreis der Kaufoption abzüglich dessen Wert zum Zeitpunkt des Verkaufs fallen (Break-Even-Kurs). Aus diesem Grund bildet man Covered Calls eher dann, wenn man von einem etwa unveränderten Kurs zum Ende der Laufzeit ausgeht, d. h. nur leicht steigender oder fallender Kurs des Basiswerts. Gewinne sind grundsätzlich auf den Ausübungspreis des Calls beschränkt, wohingegen Kursverluste im Basiswert, ab dem Break-Even-Kurs, in voller Höhe mitgetragen werden.\footnote{\url{https://de.wikipedia.org/wiki/Covered_Call}}
			\item Protective Put (1:1 Hedge): Bei einem Protective Put erwirbt man das Underlying und eine Put-Option dazu. Der Sinn ist, sich mit der Put-Option gegen ein Kursverfallsrisiko zu versichern. Ein Protective Put sichert sich einen Mindestverkaufspreis und zahlt dafür einen Aufschlag.\footnote{\url{https://de.wikipedia.org/wiki/Protective_Put}}
		\end{itemize}
		\item Spread
		\begin{itemize}
			\item Position aus ge- und verkauften Optionen derselben Klasse (nur Calls oder nur Puts)
			\item 1:1 Spreads: Anzahl der Calls (bzw. Puts) long entspricht der Anzahl an Calls (Puts) short)
			\begin{itemize}
				\item Money Spread: Verschiedene Basispreise, gleiche Optionsfristen
				\item Time Spread: Verschiedene Optionsfristen, gleiche Basispreise
				\item Diagonal Spread: Verschiedenen Basispreise, verschiedene Optionsfristen
			\end{itemize}
			\item Ratio Spread: Anzahl der Calls (bzw. Puts) long entspricht nicht der Anzahl an Calls (Puts) short)
			\item Beispiel Butterfly Spread: Bei dem Long Butterfly Spread werden zwei Calls gekauft und zwei Calls verkauft. Der erste gekaufte Call wird zu einem niedrigeren Kurs des Underlyings gekauft (in-the-money) und der zweite gekaufte Call wird zu einem höheren Kurs des Underlyings (out-of-the-money) erworben. Zusätzlich werden noch zwei Calls verkauft zum aktuellen Kurs des Underlyings (at-the-money)\footnote{\url{https://de.wikipedia.org/wiki/Optionsstrategie\#Butterfly_Spread}}
		\end{itemize}
		\item Combinations
			\begin{itemize}
				\item Position aus ge- und verkauften Optionsn verschiedener Klassen
				\item Straddle: Gleiche Basispreise, gleiche Optionsfristen
				\item Strangle: Verschiedene Basispreise, gleiche Optionsfristen
				\item Horizontal Strangle: Verschiedene Basispreise, verschiedene Optionsfristen
			\end{itemize}
	\end{itemize}
\end{itemize}



\subsubsection{Klassische Beurteilungskriterien}
Qualitative Abwägung er Chancen und Risiken anhand von Ausübungsdiagrammen.
\newline

\begin{tabularx}{\columnwidth}{|X|X|X|X|X|}
	\hline
	\textbf{\(\downarrow\)Chancen|Risiken\(\rightarrow\)} & \textbf{keine} & \textbf{begrenzt} & \textbf{fast unbegrenzt} & \textbf{unbegrenzt} \\
	\hline
	\textbf{keine} \newline sicheres Ergebnis & Zero-Bon (ohne Bonitätsrisiko) & - & - & - \\
	\hline
	\textbf{begrenzt} \newline begrenzt auf Zahlung in \(t=0\) & - & Money Spread: Butterflies & Put short & Call short \\
	\hline
	\textbf{fast begrenzt} \newline Vielfaches der Zahlung in \(t=0\) & - & Put long & Put short (X)\newline Call long (X)\newline Call short (2X) & Put long + Call short = Aktie short \\
	\hline
	\textbf{unbegrenzt} & geschenktes, heutiges Lotterielos & Call long & Put short + Call long \(\hat{=}\) Aktie & Call long + Call short auf nicht identisches Underlying \\
	\hline
\end{tabularx}


\subsection{Verteilungsfreie Wertgrenzen}
Idee: Ableitung von Preisbeziehungen zwischen verschiedenen Instrumenten (Calls, Puts, Basiswert, risikoloses Instrument, etc.) ohne explizite Annahmen bzgl. zukünftiger Kursentwicklung des Basiswertes.

\paragraph{Annahmen}
\begin{itemize}
	\item Basiswert \(S(T) \ge 0\) (beschränkte Haftung des Basiswertes)
	\item Einmalige, sichere Dividendenzahlung in \(t_1\), \(0 < t_1 < T\), in Höhe von \(D\)
	\item Beliebige nicht-dividendengeschützte Calls und Puts auf Basiswert verfügbar
\end{itemize}

\subsubsection{Untere und obere Wertgrenzen für Calls und Puts}
\begin{itemize}
	\item \textbf{Obere Wertgrenzen für Calls}
	\begin{itemize}
		\item W1) \(C^e \le C^a \le S(0)\)
	\end{itemize}
	\item \textbf{Obere Wertgrenzen für Puts}
	\begin{itemize}
		\item W2) \(P^e \le X(1+r)^{-T}\)
		\item W3) \(P^a \le X\)
	\end{itemize}
	\item \textbf{Untere Wertgrenzen für Calls}
	\begin{itemize}
		\item W4) \(C^e \ge max\big(0, S(0)-X(1+r)^{-T}-D(1+r)^{-t_1}\big)\)
		\item W5) \(C^a \ge max\big(C^e, S(0)-X\big)\)
	\end{itemize}
	\item \textbf{Untere Wertgrenzen für Puts}
	\begin{itemize}
		\item W6) \(P^e \ge max\big(0, X(1+r)^{-T}+D(1+r)^{-t_1}-S(0)\big)\)
		\item W7) \(P^a \ge max\big(P^e, X-S(0)\big)\)
	\end{itemize}
\end{itemize}

\subsubsection{Put-Call-Parität}
\begin{itemize}
	\item Intuitiv konstruiert: Risikolose Position aus Put long, Call short und Aktie short mit gleichem Basispreis und gleicher Fälligkeit
	\item Arbitragemöglichkeit, wenn die Beziehung verletzt ist
	\item Put-Call-Parität für europäische Optionen: \(C^e = P^e + S(0) - X(1+r)^{-T} - D(1+r)^{-t_1}\)
	\item Put-Call-Ungleichung für amerikanische Optionen: \(C^a - S(0) + D + X \ge P^a \ge C^a - S(0) + X(1+r)^{-T}\)
\end{itemize}


\subsection{Arbitrageorientierte Bewertung}
Ziel im Folgenden:
\begin{itemize}
	\item Quantiatives Modell zur Ermittlung von (fairen) Optionspreisen,
	\item relativ zu gegebenem Preis des Basiswertes und dessen Dynamik.
\end{itemize}

\subsubsection{Einperiodige Ökonomie}
\begin{itemize}
	\item Zwei Zeitpunkte \(t=\{0,1\}\) mit entsprechenden Umweltzuständen sowie zwei Basiswertpapiere (ein risikoloses und ein riskantes)
	\item \textbf{Basiswertpapiere}
	\begin{itemize}
		\item Risikoloses: \(P_0(0)=1\), \(P_0(1)=1+r\)
		\item Riskantes: \(P_1(0)=S\), \(P_1(1)=\begin{cases}Su& \text{falls up-Zustand} \\Sd&\text{falls down-Zustand}\end{cases}\)
		\item Ausschluss von Arbitrage: \(u>1+r>d\)
	\end{itemize}
	\item Idee: Bewertung einer Option \(P_x\) auf risikantes Wertpapier mit Ausübungswerten \(P_x(1)=\begin{cases}a^u& \text{falls up-Zustand} \\a^d&\text{falls down-Zustand}\end{cases}\)
	\item \textbf{Äquivalente Bewertungszugänge}
	\begin{enumerate}
		\item Duplikationsprinzip
		\begin{itemize}
			\item Konstruktion eines Portefeuilles aus riskantem Wertpapier (Underlying der Option) und risikolosesm Wertpapier, das Ausübungswerte der Option repliziert
			\item Aus dem Gesetz des einen Preises folgt der Wert des Duplikationsportefeuille, welches dem gesuchten Optionswert entspricht
		\end{itemize}
		\item Hedgingprinzip
		\begin{itemize}
			\item Konstruktion eines risikolosen Portefeuilles aus riskantem Wertpapier (Underlying der Option) und der Option
			\item Aus dem Gesetz des einen Preises folgt, dass das Hedgeportefeuille eine risikolose Rendite erzielen muss, welchem dem gesuchten Optionswert entspricht
		\end{itemize}
		\item Duplikation durch Arrow-Debreu-Wertpapiere und risikoneutrale Bewertung
		\begin{itemize}
			\item Arrow-Debreu-Wertpapiere: Zustandspreise und risikoadjustierte (risikoneutrale) Wahrscheinlichkeiten
			\item Risikoneutrale Bewertung unter Verwendung risikoadjustierter Wahrscheinlichkeiten (Normierung der Zustandspreise)
		\end{itemize}
	\end{enumerate}
	\item Verallgemeinerung auf \(m\) Zustände mit weiterhin zwei Zeitpunkten möglich
\end{itemize}

\subsubsection{Binomialmodell}
\begin{itemize}
	\item Grundmodell: Diskrete Handelszeitpunkte sowie Zustandsraum mit zugehörigen Wahrscheinlichkeiten
	\item Zwei Basiswertpapiere: Risikoloses Geldmarktkonto (Money Market Account) und riskantes Wertpapier ohne zwischenzeitliche Auszahlungen
	\item Bezeichnung im Folgenden: \(P_x(t)^i\) (Preis des Wertpapiers \(P_x\) im Zeitpunkt \(t\) bei \(i\) Aufwärtsschritten)
\end{itemize}

\subsubsection{Zeitstetige Bewertung und das Black und Scholes Modell}
\begin{itemize}
	\item Black-Scholes-Formel als Grenzwert des Binomialmodells
	\item Modellökonomie, die jederzeit Wertpapierhandel zulässt, wobei Wertpapiere beliebige Werte annehmen können. Wertpapierpreise folgen zeit- und zustandsstetigen stochastischen Prozessen
	\item \textbf{Wichtige Eigenschaften stochastischer Prozesse}
	\begin{itemize}
		\item Martingaleigenschaft: Stochastischer Prozess, der über einen bedingten Erwartungswert definiert wird und im Mittel fair ist\footnote{\url{https://de.wikipedia.org/wiki/Martingal}}
		\item Markoveigenschaft: Eine Markow-Kette ist darüber definiert, dass auch durch Kenntnis einer nur begrenzten Vorgeschichte ebenso gute Prognosen über die zukünftige Entwicklung möglich sind wie bei Kenntnis der gesamten Vorgeschichte des Prozesses\footnote{\url{https://de.wikipedia.org/wiki/Markow-Kette}}
	\end{itemize}
	\item \textbf{Fundamentale Bausteine zur Modellierung von Wertpapierpreisen}
	\begin{itemize}
		\item Normale Kursänderung: Mit kürzerem Beobachtungszeitraum werden Kursveränderungen kleiner (Wiener Prozess, im Folgenden verwendet)
		\item Seltene Kursänderung: Mit kürzerem Beobachtungszeitraum wird Wahrscheinlichkeit der Veränderung kleiner, Größe der Veränderung ist fest (Poisson Prozess)
	\end{itemize}
	\item \textbf{Wiener Prozess}
	\begin{itemize}
		\item Definition: \(w(0)=0\), \(w(t)\) hat unabhängige Zustände, \(w(t)-w(s),t>s\) ist normalverteilt mit \(w(t)-w(s)\sim\mathcal{N}(0,t-s)\)
		\item Lässt sich als Grenzwert einer Summe diskreter Zufallsvariablen mit gleichem Wert und gleicher Wahrscheinlichkeit auffassen (siehe Random Walk\footnote{\url{https://de.wikipedia.org/wiki/Random_Walk}}, da Martingal)
		\item Verallgemeinert mit Aktienkurs \(x(t)\)
		\begin{itemize}
			\item Stochastische Differentialgleichung, welche die zeitliche Entwicklung von \(x\) beschreibt
			\item Deterministische Entwicklung (Drift \(\alpha\)) wird von stochastischem Term (Volatilität \(\sigma\)) überlagert
			\item \(dx(t) = \alpha(x,t)\cdot dt + \sigma(x,t)\cdot dw\)
			\item Berechnung mit Hilfe von Itô's Lemma
		\end{itemize}
	\end{itemize}
	\item \textbf{Die Black-Scholes-Bewertungsgleichung}
	\begin{itemize}
		\item Annahmen: Zeitstetiger Handel im Intervall \(\lbrack 0,T\rbrack\), vollkommener Kapitalmarkt, ein risikoloses Papier und eine dividendenlose Aktie mit Kursprozess, Arbitragefreiheit
		\item Kursprozess der dividendenlosen Aktie: \(d(S) = \mu \cdot S \cdot dt + \sigma \cdot S \cdot dw\), wobei \(\mu,\sigma > 0\)
		\item Ableitung der Bewertungsgleichung durch zeitstetige Replikation
		\begin{itemize}
			\item Grundüberlegung: Option wird durch die selbe Unsicherheit getrieben wie die Aktie \(\rightarrow\) durch geeignete Portefeuillebildung kann lokal risikoloses Portefeuille erzeugt werden
			\item \(P\) selbstfinanzierendes Portefeuille mit \(\theta_1\) Aktien und \(\theta_2\) Optionen. Portefeuillewert entspricht \(\theta_1 \cdot S + \theta_2 \cdot C\)
			\item Löst man den Preisprozess des Portefeuilles \(dP = \theta_1dS + \theta_2dC\) mit Hilfe von Itô's Lemma und setzt \(\theta_1 = -\theta_2C_s\) kann man mit Hilfe von No-Arbitrage (gleichsetzen) ein lokal risikoloses Protefeuille erzeugen
			\item Daraus ergibt sich die undamentale Bewertungsgleichung: \(C_t + C_sSr + \frac{1}{2}C_{SS}\sigma^2S^2 = Cr\)
		\end{itemize}
	\end{itemize}
\end{itemize}



\section{Appendix A: Exkurse}

\subsection{Stetige Zinsrechnung}

Die stetige Verzinsung ist ein Sonderfall der unterjährigen Verzinsung mit Zinseszinsen, bei der die Anzahl der Zinsperioden gegen unendlich strebt. Der Zeitraum der einzelnen Zinsperiode geht also gegen \(0\).\footnote{\url{https://de.wikipedia.org/wiki/Zinsrechnung\#Stetige_Verzinsung}}

Variation der Verzinsungsperiode: \(r_m\) Zinssatz p.a. bei \(m\) Verzinsungsperioden pro Jahr, lineare Umrechung von Jahreszinssatz in Periodenzinssatz

\[\Big(1+\frac{r_m}{m}\Big)^{mT} \longrightarrow e^{rT}\]


\section{Appendix B: Bezeichnungen und Formeln}

\subsection{Bezeichnungen}
\begin{itemize}
	\item Kassapreis des Basiswerts zum Zeitpunkt \(t\): \(S(t)\)
	\item Terminpreis in \(t\) für den Kauf des Basiswerts in \(T\): \(f(t,T)\)
	\item Absicherungsniveau: Zukünftiger Kaufkurs, also \(f(0,T)\)
	\item Glattstellen: Neutralisieren einer Transaktion durch Einnehmen der Gegenposition zur Risikominimierung
	\item Stückzinsen: Zinsausgleichszahlung an bisherigen Inhaber der Anleihe, wenn Kauf zwischen Kuponterminen. Kupon wird dabei anteilig nach Besitzzeit zwischen Neu- und Altbesitzer geteilt
	\item Callwert zum Zeitpunkt \(t\): \(C(t)\)
	\item Putwert zum Zeitpunkt \(t\): \(P(t)\)
	\item Basispreis, Strike price: \(X\)
	\item Europäischer Call- bzw. Put-Wert: \(C^e, P^e\)
	\item Amerikanischer Call- bzw. Put-Wert: \(C^a, P^a\)
\end{itemize}


\subsection{Formeln}
\begin{itemize}
	\item Terminpreis in \(t\) für den Kauf des Basiswerts in \(T\): \(f(t,T) = S(t)\cdot e^{r\cdot (T-t)}\)
	\item Ertragsloser Basiswert mit erwarteter Rendite \(q\): \(f(t,T) = S(0) \cdot e^{(r-q)\cdot T-t}\)
	\item Terminpreis = Kassapreis + Haltekosten (Kreditzinsen, Lagerkosten, etc.) - Halteerträge (Dividenden, Zinserträge, etc.)
	\item Basis = Forwardpreis - Kassapreis
	\item Terminpreis in \(t=0\) für den Kauf des Basiswerts in \(T\) mit Halteertrag: \(f(0,T) = (S(0)-X(0))\cdot e^{rT}\)
	\item Kontinuierlicher Zins vs. Zins p.a.: \(r_c = m \cdot ln\Big(1+\frac{r_m}{m}\Big)\), \(r_m\) bereits auf das komplette Jahr hochgerechnet
	\item Wert eines Forward: \(W_K(t,T) = \Big(f(t,T)-f(0,T)\Big) \cdot e^{r\cdot (T-t)}\)
	\item \textbf{Arbitragefreie Bewertung von Optionen}
	\begin{itemize}
		\item Einperiodige Ökonomie
		\begin{enumerate}
			\item Duplikationsprinzip
			\begin{itemize}
				\item Anzahl der riskanten Wetpapiere: \(\Delta=\frac{a^u-a^d}{S^u-S^d}\)
				\item Anzahl der risikolosen Wertpapiere: \(B=\frac{ua^d-da^u}{(u-d)\cdot(1+r)}\)
				\item Gesetz des einen Preises: \(P_x(0)=\Delta S+B=\Big(\frac{1+r-d}{u-d}\cdot a^u + \frac{u-(1+r)}{u-d}\cdot a^d\Big)\cdot\frac{1}{1+r}\)
			\end{itemize}
			\item Hedging-Prinzip
			\begin{itemize}
				\item Da risikolos gilt immer: \(\Delta S^u - a^u = \Delta S^d - a^d\)
				\item \(P_x(0) = \Delta S - \frac{S^d - a^d}{1+r}\)
			\end{itemize}
			\item Duplikation durch Arrow-Debreu-Wertpapiere (Einperiodige Ökonomie)
			\begin{itemize}
				\item \(q=\frac{S(0)\cdot(1+r)-S_d(1)}{S_u(1)-S_d(1)}\)
				\item \(C^a=C^e=\frac{1}{1+r}\cdot\Big(q\cdot W_c^u(1)+(1-q)\cdot W_c^d(1)\Big)\)
				\item \(P^e=\frac{1}{1+r}\cdot\Big(q\cdot W_p^u(1)+(1-q)\cdot W_p^d(1)\Big)\)
				\item \(P^a=max\Big\{P^e,X-S(0)\Big\}\)
			\end{itemize}
		\end{enumerate}
		\item Binomialmodell
		\begin{itemize}
			\item Rückwärtsrechnung, beginnend bei \(t=T\)
			\item \(q=\frac{S(t)\cdot(1+r)-S_d(t)}{S_u(t)-S_d(t)}\)
			\item \(C^a=C^e=\frac{1}{1+r}\cdot\Big(q\cdot W_c^u(t)+(1-q)\cdot W_c^d(t)\Big)\)
			\item \(P^e=\frac{1}{1+r}\cdot\Big(q\cdot W_p^u(t)+(1-q)\cdot W_p^d(t)\Big)\)
			\item \(P^a=max\Big\{P^e,X-S(t)\Big\}\)
			\item \(u=e^{\sigma\cdot\sqrt{\frac{T}{n}}}\), \(d=\frac{1}{u}\)
		\end{itemize}
		\item Black-Scholes-Modell
		\begin{itemize}
			\item Aktienkurs als Wiener Prozess: \(dx(t) = \alpha(x,t)\cdot dt + \sigma(x,t)\cdot dw\)
			\item Deterministischer Prozess (\(\sigma(x,t) \equiv 0\)): \(df=f_t\cdot dt + f_x\cdot dx\)
			\item Erweiterung auf Itô Prozess: \(df=f_t\cdot dt + f_x\cdot dx + \frac{1}{2} f_{xx}(dx)^2\)
		\end{itemize}
	\end{itemize}
\end{itemize}
