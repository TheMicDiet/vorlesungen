\chapter{Netzsicherheit: Architekturen und Protkolle}

Zusammenfassung der Vorlesung "`Netzsicherheit: Architekturen und Protokolle"' aus dem Sommersemester 2016.\footnote{\url{https://telematics.tm.kit.edu/ss2016_2928.php}}

\section{Einführung}
\begin{itemize}
	\item Smarte Welt - alles vernetzt. Vorteile beispielsweise: Bessere Integration erneuerbarer Energien, bessere Organisation des Verkehrsm "`assistiertes Leben"'
	\item Vernetzte Daten: Sensoren übermitteln die erfassten Daten an einen zentralen Datenspeicher im Internet
	\item Problem: Systeme können (beispielsweise über das Internet) angegriffen werden, Daten können von unbefugten Dritten mitgelesen werden
	\item Alternative: Vollständig verteiltes System. Probleme: Vertrauensbasis? Kontrolle? Nachvollziehbarkeit? Zuverlässigkeit?
	\item Alternative: Isoliertes System? Kein Zugriff von außen, daher theoretisch sicher. Praktisch existiert immer eine Verbindung nach außen, z.B. zum installieren von Updates
\end{itemize}


\subsection{Security vs. Safety}

\subsubsection{Begriffsdefinitionen}
\begin{itemize}
	\item (IT-)System: Gesamtheit von Komponenten, die zusammenwirken, um eine bestimmte Funktionalität zu erfüllen
	\item Komponente: Bestandteil eines Systems, das eine Teilfunktion dessen realisiert und über Schnittstellen mit anderen Komponenten kommuniziert
	\item Güter: Ressourcen die für mindestens einen Akteur einen (subjektiven) Wert besitzen
	\item Schutzziel: Anforderungen an eine Komponente oder ein System, um Güter vor Bedrohungen zu schützen
	\item Angreifermodell: Beschreibt die Fähigkeiten eines Angreifers, Angriffe auf ein System durchzuführen (beispielsweise Lokalität, Werkzeuge, kryptografische Fähigkeiten)
	\item \textbf{Safety}
	\begin{itemize}
		\item Zustand des Geschütztsein von schützenswerten Gütern vor bestimmten Gefahren
		\item Ist-Funktionalität von Komponenten stimmt mit der Soll-Funktionalität überein
	\end{itemize}
	\item \textbf{Security}
	\begin{itemize}
		\item Angriffssicherheit
		\item Bedrohung durch böswilligen Angreifer
		\item Beispielsweise Schutz der Integrität von Informationen
	\end{itemize}
\end{itemize}


\subsection{Schutzziele}
\begin{itemize}
	\item \textbf{Vertraulichkeit}
	\begin{itemize}
		\item Ein System bewahrt Vertraulichkeit, wenn es keine unautorisierte Informationsgewinnung ermöglicht
		\item Bausteine: Symmetrische oder asymmetrische Verschlüsselung
	\end{itemize}
\end{itemize}


\subsection{Typische Angriffe}


\subsection{Schutzmechanismen und Bausteine}



\section{Schlüsselaustausch}



\section{Vertrauensmodelle}



\section{Authentifizierung}



\section{Kerberos}



\section{Zugangsschutz}



\section{IPsec}



\section{TLS}



\section{Internetdienste}



\section{Privatsphäre}
