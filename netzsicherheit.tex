\chapter{Netzsicherheit: Architekturen und Protkolle}

Zusammenfassung der Vorlesung "`Netzsicherheit: Architekturen und Protokolle"' aus dem Sommersemester 2016.\footnote{\url{https://telematics.tm.kit.edu/ss2016_2928.php}}

\section{Einführung}
\begin{itemize}
	\item Smarte Welt - alles vernetzt. Vorteile beispielsweise: Bessere Integration erneuerbarer Energien, bessere Organisation des Verkehrsm "`assistiertes Leben"'
	\item Vernetzte Daten: Sensoren übermitteln die erfassten Daten an einen zentralen Datenspeicher im Internet
	\item Problem: Systeme können (beispielsweise über das Internet) angegriffen werden, Daten können von unbefugten Dritten mitgelesen werden
	\item Alternative: Vollständig verteiltes System. Probleme: Vertrauensbasis? Kontrolle? Nachvollziehbarkeit? Zuverlässigkeit?
	\item Alternative: Isoliertes System? Kein Zugriff von außen, daher theoretisch sicher. Praktisch existiert immer eine Verbindung nach außen, z.B. zum installieren von Updates
\end{itemize}


\subsection{Security vs. Safety}

\subsubsection{Begriffsdefinitionen}
\begin{itemize}
	\item (IT-)System: Gesamtheit von Komponenten, die zusammenwirken, um eine bestimmte Funktionalität zu erfüllen
	\item Komponente: Bestandteil eines Systems, das eine Teilfunktion dessen realisiert und über Schnittstellen mit anderen Komponenten kommuniziert
	\item Güter: Ressourcen die für mindestens einen Akteur einen (subjektiven) Wert besitzen
	\item Schutzziel: Anforderungen an eine Komponente oder ein System, um Güter vor Bedrohungen zu schützen
	\item Angreifermodell: Beschreibt die Fähigkeiten eines Angreifers, Angriffe auf ein System durchzuführen (beispielsweise Lokalität, Werkzeuge, kryptografische Fähigkeiten)
	\item \textbf{Safety}
	\begin{itemize}
		\item Zustand des Geschütztsein von schützenswerten Gütern vor bestimmten Gefahren
		\item Ist-Funktionalität von Komponenten stimmt mit der Soll-Funktionalität überein
	\end{itemize}
	\item \textbf{Security}
	\begin{itemize}
		\item Angriffssicherheit
		\item Bedrohung durch böswilligen Angreifer
		\item Beispielsweise Schutz der Integrität von Informationen
	\end{itemize}
\end{itemize}


\subsection{Schutzziele}
\begin{itemize}
	\item \textbf{Vertraulichkeit}
	\begin{itemize}
		\item Ein System bewahrt Vertraulichkeit, wenn es keine unautorisierte Informationsgewinnung ermöglicht
		\item Bausteine: Symmetrische oder asymmetrische Verschlüsselung
	\end{itemize}
	\item \textbf{Integrität}
	\begin{itemize}
		\item Starke Integrität: Es ist nicht möglich, Daten unautorisiert zu manipulieren
		\item Schwache Integrität: Es ist nicht möglich, Daten unautorisiert \textit{unbemerkt} zu manipulieren. Manipulation ist in vielen Fällen nicht verhinderbar, sollte dann aber nicht unbemerkt bleiben
		\item Bausteine: Tamper proof Module, Message Authentication Codes (MAC)
	\end{itemize}
	\item \textbf{Authentizität}
	\begin{itemize}
		\item Echtheit von Subjekten und/oder Daten
		\item Bausteine: Zertifikate, Signaturen, gemeinsames Geheimnis
	\end{itemize}
\end{itemize}


\subsection{Typische Angriffe}
\begin{itemize}
	\item \textbf{Angreifermodell}
	\begin{itemize}
		\item Idee: Klassifikation von Angreifern nach Ressourcen/Motivation/Fähigkeiten zur Bestimmung des Sicherheitsniveaus (Gegen welche Art von Angreifer will/kann ich mich schützen?)
		\item Dolev-Yao-Angreifer: Angreifer ist omnipräsent, kann Dateneinheiten erzeugen/versenden/modifizieren, kann allerdings nicht ver- oder entschlüsseln ohne den Schlüssel zu kennen (Angreifer einspricht "`Outsider"')
	\end{itemize}
	\item \textbf{Systematische Einordnung von Angriffen}
	\begin{itemize}
		\item Passiv: Unautorisierte Informationsgewinnung \(\rightarrow\) Vertraulichkeit
		\item Aktiv: Unautorisierte Manipulation \(\rightarrow\) Integrität/Verfügbarkeit
		\item Typische Angriffstechniken: Abhören/Zwischenschalten (beispielsweise MitM)/Manipulieren/Unterdrücken/Einfügen (beispielsweise DoS)/Replay
	\end{itemize}
\end{itemize}


\subsection{Schutzmechanismen und Bausteine}
\begin{itemize}
	\item \textbf{Kryptografische Bausteine}
	\begin{itemize}
		\item Symmetrische oder asymmetrische Verschlüsselung
		\item Integritätssicherung durch kryptografische Hashfunktion oder digitale Signatur
	\end{itemize}
	\item \textbf{Zertifikate}
	\begin{itemize}
		\item Authentifizierung eines Sachverhalts, den man nicht selbst überprüfen kann durch vertrauenswürdige Dritte (CA)
		\item Digitales Dokument, in dem eine Instanz einen bestimmten Sachverhalt mittels digitaler Signatur bestätigt
	\end{itemize}
	\item \textbf{Authentifizierung}
	\begin{itemize}
		\item Dient der Überprüfung, ob ein Kommunikationspartner tatsächlich derjenige ist, der er vorgibt zu sein
		\item Möglichkeiten: (Kombination aus) Besitz/Wissen/Biometrisches Merkmal
		\item Mechanismen und Bausteine
		\begin{itemize}
			\item Passwörter oder Passwort-Hashes: Authentifikation durch Nachweis eines Geheimnissen. Nachteile u.a.: Passwortliste notwendig (Ziel für Angreifer), Passwort muss übertragen werden (Vertraulichkeit eventuell gefährdet)m oft schlechte Wahl der Passwörter
			\item Challenge-Response-Authentifizierung: Vergleichbare Probleme wie bei Passwörtern
		\end{itemize}
	\end{itemize}
\end{itemize}



\section{Schlüsselaustausch}

\subsection{Problemstellung}
\begin{itemize}
	\item Wie können Schlüssel sicher über einen ungesicherten Kanal ausgetauscht werden?
	\item \textbf{Statische Ansätze}
	\begin{itemize}
		\item Persönliche Übergabe: Sehr einfach und mit automatischer Authentifizierung, allerdings persönliches Treffen notwendig und schlechte Skalierung
		\item Hinterlegen des Schlüsselmaterials bei einer vertrauenswürdigen Instanz
		\begin{itemize}
			\item Vorteile: Einfaches Verfahren, kein persönliches Treffen notwendig, weniger Schlüssel bei den Kommunikationspartnern zu speichern
			\item Nachteile: Sicherer Kanal erforderlich, Schlüssel nur indirekt authentifiziert, zentrale Infrastuktur notwendig (Gefahr durch Ausfälle oder Korruptions)
		\end{itemize}
	\end{itemize}
	\item \textbf{Dynamische Ansätze}
	\begin{itemize}
		\item Nutzung asymmetrischer Verfahren (beispielsweise RSA)
		\item Austausch eines geheimen, symmetrischen Sitzungsschlüssels über einen nicht vertrauenswürdigen Kanal
		\begin{itemize}
			\item Verschlüsselung des Sitzungsschlüssel mit öffentlichem Schlüssel des Kommunikationspartners
			\item Vorteile: Kein persönliches Treffen oder zentrale Infrastruktur notwendig, Sitzungsschlüssel sind nach Prüfung der öffentlichen Schlüssel authentifiziert
			\item Nachteile: Sitzungsschlüssel an langlebiges Geheimnis gebunden (keine Perfect Forward Secrecy), rechenintensiv
		\end{itemize}
		\item Diffie-Hellman-Verfahren
		\begin{itemize}
			\item Vorteile: Kein persönliches Treffen oder zentrale Infrastruktur notwendig, dynamische Aushandlung \(\rightarrow\) Perfect Forward Secrecy
			\item Nachteile: Schlüssel nicht authentifiziert, sehr rechenintensiv
		\end{itemize}
	\end{itemize}
\end{itemize}


\subsection{Diffie-Hellman}
\begin{itemize}
	\item Problemstellung: Wie können Schlüssel sicher über einen ungesicherten Kanal ausgetauscht werden?
	\item Vorgehen: A und B einigen sich auf gemeinsame Zufalls(prim)zahlen und leiten daraus per Einwegfunktion individuelle Geheimnissen ab, die zur gegenseitigen Echtheitsprüfung verwendet werden. Die Sicherheit beruht auf der "`schwierigen"' Rückberechnung durch die Einwegfunktion
	\item Vorteile: Keine Infrastruktur notwendig, geheime Zufallszahlen werden nach dem Schlüsselaustausch gelöscht \(\rightarrow\) Perfect Forward Secrecy
	\item Nachteile: Anonymer Schlüsselaustausch \(\rightarrow\) keine Authentifizierung der Teilnehmer, MitM-Angriffe möglich, rechenintensiv (und damit anfällig für DoS-Angriffe)
\end{itemize}


\subsection{Bausteine des Schlüsselaustauschs}

\subsubsection{Schlüsselaustauschprotokolle: Bedrohungen}
\begin{itemize}
	\item Man-in-the-Middle-Angriffe: Schlüsselaustausch wird unwissentlich mit dem Angreifer ausgeführt
	\item Replay-Attacken: Wiedereinspielungsangriff von zuvor aufgezeichneten Nachrichten
	\item Denial-of-Service-Angriffen
	\item Downgrade-Attacken: Löschen von starken Algorithmen aus Liste der unterstützten Verfahren zugunsten von veralteten (potentiell schwächeren) Verfahren
	\item Missbräuchliche Schlüsselhinterlegung bei einer vertrauenswürdigen Organisation
\end{itemize}

\subsubsection{Bausteine}
\begin{itemize}
	\item \textbf{Perfect Forward Secrecy}
	\begin{itemize}
		\item Bedingung: Angreifer kann die Kommunikation auch dann nicht entschlüsseln, wenn er die komplette Kommunikation aufgezeichnet hat und das langlebige Geheimnis entwedet (Einbruch in Endsysteme)
		\item Maßnahmen: Sitzungsschlüssel muss von langlebigem Geheimnis unabhängig sein, alle Sitzungsinformationen müssen nach Beendigung der Sitzung gelöscht werden, langlebiges Geheimnis muss periodisch erneuert werden
		\item Beispiel: DH-Austausch mit Authentifizierung
	\end{itemize}
	\item \textbf{Schutz der Identitäten}
	\begin{itemize}
		\item Problem: Passiver Angreifer kann Identitäten der Kommunikationspartner abhören
		\item Lösung: Zunächst anonymer DH-Schlüsselaustauch, danach Übertragung der Identität
	\end{itemize}
	\item \textbf{Dynamische Wahl der Verfahren}
	\begin{itemize}
		\item Dynamische Wahl der genutzten Sicherheitsmechanismn zur Verbesserung der Interoperabilität
		\item Vorteile: Einfache Migration zu kryptographisch stärkeren Verfahren sowie Ausschluss gebrochener Verfahren
		\item Probleme: Komplexität des Protokolls (Wie werden Sicherheitsmechanismen beschrieben und welche Kombinationen sind zulässig?), Downgrade-Angriff
	\end{itemize}
	\item \textbf{Verhinderung von Downgrade-Angriffen\footnote{\url{http://www.golem.de/news/browser-downgrade-angriffe-auf-tls-1309-101305.html}}}
	\begin{itemize}
		\item Ziel des Angreifers: Durch gezielte Verbindungsstörungen dafür sorgen, dass der Client eine Verbindung mit einer älteren Protokollversion durchführt
		\item Anfällig sind beispielsweise Browser, die ältere Protokollversionen als Fallback nutzen: Fast alle gängigen Browser - Firefox, Chrome, Safari und der Internet Explorer - führen bei Verbindungsabbrüchen ein Downgrade auf SSLv3 durch
		Lösungsansatz: Erkennen von Manipulationsversuchen durch Integritätsschutz (beispielsweise HMAC) aller gesendeten Nachrichten
	\end{itemize}
	\item \textbf{Einschränkung von DoS-Angriffen}
	\begin{itemize}
		\item Ziel: Keine Durchführung von rechenintensiven Operationen, solange nicht klar ist, dass die Absenderadresse nicht gefälscht ist
		\item Möglichkeiten zur Echtheitsprüfung des Anfragenden
		\begin{itemize}
			\item Token/Cookie: Enthält zufällige Informationen des Angefragten. Wird es zurückgesendet, ist mit oher Wahrscheinlichkeit die Absenderadresse nicht gefälscht (beispielsweise durch IP-Spoofing)
			\item Puzzle: Stellen einer rechenintensiven Aufgabe an den Anfragenden. Keine lokale Zustandshaltung beim Angefragten
		\end{itemize}
		\item Anforderungen an Tokens/Cookies: Aktualität, Eindeutigkeit, Unverhersagbarkeit, einfache Erzeugung (hoher Aufwand würde DoS-Angriff begünstigen), leichte Verifikation
		\item Sitzungswiederaufnahme
		\begin{itemize}
			\item Problem: Initialer Schlüsselaufwand häufig teuer, Verbindungsabbruch bedingt erneuten Schlüsselaustausch
			\item Zustandsbehafteter Ansatz: Zustand nach Schlüsselaustausch wird bei einem Kommunikationspartner gespeichert und kann wiederhergestellt werden (Vgl. TLS-Session-Resumption)
			\item Zustandslos: Zustand wird komplett in geschütztes Ticket kodiert und ausgelagert. Kann bei Wiederaufnahme der Sitzung vorgelegt werden (Vgl. Tickets bei Kerberos)
		\end{itemize}
	\end{itemize}
\end{itemize}


\subsection{Digitale Zertifikate}
\begin{itemize}
	\item Sichere Zuordnung von öffentlichen Schlüsseln und Identität der Kommunikationspartner
	\item Lösung: Verwendung von Zertifikaten. Authentifizierung hier durch vertrauenswürdige Dritte und Bestätigung durch digitale Signatur (Vgl. Personalauswand/Reisepass)
\end{itemize}



\section{Vertrauensmodelle}

\subsection{Motivation}
\begin{itemize}
	\item Idee: Absichern von asymmetrisch verschlüsselter Kommunikation mit Hilfe von digitalen Zertifikaten. Zentrale Frage: Wer erstellt die vertrauenswürdige Instanz?
	\item \textbf{Neue Probleme}
	\begin{itemize}
		\item Vertrauen in die CA, bzw. deren Integrität
		\item Gültigkeit der Zertifikate
		\item Authentizität des öffentlichen Schlüssels der CA
		\item Konflikte, wenn mehrere CAs die selbe Entität signieren
		\item Wer steht an der Spitze der CA(-Kette)?
	\end{itemize}
	\item \textbf{Widerruf digitaler Zertifikate}
	\begin{itemize}
		\item Widerruf manchmal notwendig, beispielsweise wenn der private Schlüssel verloren oder korrumpiert worden ist
		\item CA stellt signierte \textit{Certificate Revocation List} zur Verfügung
	\end{itemize}
\end{itemize}


\subsection{Infrastrukturen}

\subsubsection{Public Key Infrastructure (PKI)}
\begin{itemize}
	\item Zentrale Infrastructur zum Management von ID-Zertifikaten. Ermöglicht somit Authentifizierung öffentlicher Schlüssel
	\item \textbf{Anforderungen}
	\begin{itemize}
		\item Vertrauenswürdigkeit
		\item Sicherheit interner Abläufe und der Signaturschlüssel der CA
		\item Effiziens, Skalierbarkeit, Komfort für den Benutzer
	\end{itemize}
	\item \textbf{Elemente}
	\begin{itemize}
		\item Benutzer: Person oder Serverinstanz
		\item Registration Authority (RA): Implementiert die administrativen Aspekte der PKI, Schnittstelle zwischen Benutzer und CA
		\item Certification Authority (CA): Führt die Zertifizierungen durch und ist für den Schutz der eigenen privaten Schlüssel zuständig
		\item Verzeichnisdienst (Directory): Verwaltet die Zertifikate (beispielsweise LDAP) und publiziert die Widerrufslisten
	\end{itemize}
\end{itemize}

\subsubsection{Privilege Management Infrastructure (PMI)}
\begin{itemize}
	\item \textbf{Methoden zur Autorisierung}
	\begin{itemize}
		\item Access Control List (ACL): Definiert, wer auf eine Ressource zugreifen kann
		\item Discretionary Access Control: Individuelle, feingranulare pro Benutzer
		\item Mandantory Access Control: Benutzer werden klassifiziert und bekommen klassenbezogene Zugirffsrechte
		\item (Hierachical) Role-based Access Control: Benutzer werden Rollen zugewiesen, die über entsprechende Rollen verfügen. Ggf. vererbt
		\item Attributzertifikat: Attestieren Identitäten ein bestimmtes Privileg (als Attribut implementiert). Zeitlich beschränkt und kann per CRL zurückgerufen werden
	\end{itemize}
	\item PMIs realisieren Autorisierung auf Basis von Attributzertifikaten. Aufbau mit PKI vergleichbar
	\item \textbf{Aufbau einer PMI}
	\begin{itemize}
		\item Attribute Authority (AA, vgl. CA): vergibt Zugriffsrechte und zertifiziert diese
		\item Source of Authority (SOA, vgl. Root CA): Oberste Attribute Autority, zertifiziert alle weiteren Privilegien
	\end{itemize}
\end{itemize}


\subsection{Vertrauen/Vertrauensmodelle}
\begin{itemize}
	\item Vertrauen: Normal im Alltagsleben, subjektiv, unscharf, gerichten, kontextgebunden, risikoabhängig, etc.
	\item \textbf{Vertrauensmodell (Trust Model)}
	\begin{itemize}
		\item Beschreibt, welchen Zertifikaten ein Benutzer trauen kann, wie Vertrauen hergestellt wird und dieses Verhalten eingeschränkt/kontrolliert werden kann
		\item Vertrauensanker (Trust Anchor): Ausgangspunkt einer Zertifizierungskette (beispielsweise eine Root CA)
		\item Modelle
		\begin{itemize}
			\item Single-CA
			\begin{itemize}
				\item Eine CA erstellt alle Zertifikate
				\item Vorteil: Nur ein Vertrauensanker erleicht die Validierung
				\item Nachteile: Globales Vertrauen notwendig, Monopolstellung, Kompromittierung des CA-Schlüssels hat globale Konsquenzen, Skalierung, Single-Point-of-Failure
			\end{itemize}
			\item Oligarchie von CAs
			\begin{itemize}
				\item Zertifizierung durch mehrere CAs (Distributed Trust Architecture)
				\item Vorteile: Keine Monopolstellung, Kompromittierung hat begrenzte Auwirkung
				\item Nachteile: Initiale Prüfung sowie Validierung durch mehrere CAs, mehrere CA-Schlüssel müssen geschützt werden
			\end{itemize}
		\end{itemize}
		\item Transitivität von Vertrauen
		\begin{itemize}
			\item Transitives Vertrauen für komplexere Vertrauensmodelle notwendig
			\item Mathematische Definition: Wenn \(A\) Vertrauen in \(B\) (und dessen Zertifizierungen) hat und \(B\) Vertrauen in \(C\) (und dessen Zertifizierungen), so kann \(A\) auch Vertrauen in \(C\) (und dessen Zertifizierungen) haben
			\item Anstatt einem einzigen Zertifikat zu vertrauen werden jetzt Zertifikatsketten zwischen Vertrauensanker und Endpunkt aufgebaut und jede Stufe separat validiert
		\end{itemize}
		\item Transitive Modelle
		\begin{itemize}
			\item Oligarchie von CAs mit Delegierung
			\begin{itemize}
				\item CAs können untergeordnete CAs (Sub-CAs) einsetzen. Dadurch ergeben sich Zertifikatsketten
				\item Vorteile: Kompromittierung von CA-Schlüsseln hat begrenzten Wirkungsbereich, Skalierung
				\item Nachteile: Höhere CA-Schlüsselzahl notwendig, Validierung aufwendiger
			\end{itemize}
			\item Top-Down
			\begin{itemize}
				\item Single-CA mit Delegation und Einschränkunge der Delegation auf Teilbereiche eines hierarchischen Namensraums (beispielsweise DNS oder X.500)
				\item Vorteile: Kompromittierung von CA-Schlüsseln hat begrenzten Wirkungsbereich, Skalierung, kontrollierte Delegation
				\item Nachteile: Höhere CA-Schlüsselzahl notwendig, Validierung aufwendiger, immer Validierung des ganzen Pfades
			\end{itemize}
			\item Anarchie
			\begin{itemize}
				\item Jeder Benutzer fungiert als CA und je nach Bedarf (transitiv) eingesetzt werden (beispielsweise PGP)
				\item Vorteil: Auswirkung bei Kompromittierung beschränkt
				\item Nachteile: Alle Schlüssel sind CA-Schlüssel, Skalierbarkeit (hohe Anzahl Schlüssel \(\rightarrow\) Pfadfindung schwer, da nicht eindeutig), keine einheitliche Zertifizierungspolitik (Transitivität von Vertrauen problematisch), Zertifizierungen schwer kontrollierbar/einschränkbar
			\end{itemize}
		\end{itemize}
	\end{itemize}
\end{itemize}


\subsection{Beispielprotokolle}

\subsubsection{X.509}

\subsubsection{OCSP/SCVP}



\section{Authentifizierung}



\section{Kerberos}



\section{Zugangsschutz}



\section{IPsec}



\section{TLS}



\section{Internetdienste}



\section{Privatsphäre}
