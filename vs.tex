\chapter{Virtuelle Systeme}

Zusammenfassung der Vorlesung "`Virtuelle Systeme"' aus dem Wintersemester 2016.\footnote{\url{https://os.itec.kit.edu/deutsch/3257_3261.php}}

\section{Einführung}
\begin{itemize}
	\item Abstraktion: Versteckt Implementierungsdetails niedrigerer Ebenen und stellt vereinfachte Interfaces zu Ressourcen zur Verfügung \(\rightarrow\) vereinfacht das Design auf höheren Ebenen. Beispiel: Dateien als Abstraktion einer Festplatte mit Interface (\texttt{open, read, write})
	\item \textbf{Interfaces}
	\begin{itemize}
		\item Instruction Set Architectur (ISA): Betriebssystemunabhängiges Interface für den Hardwarezugriff; beispielsweise \texttt{IA-32}
		\item Application Binary Interface (ABI): Interface für den Hardwarezugriff, das vom Betriebssystem zur Verfügung gestellt wird; beispielsweise \textit{system calls}
		\item Nachteile: Eventuell schlechte Portabilität, da höhere Ebenen von entsprechenden Interfaces abhängig sind; beispielsweise ist Software an bestimmte ISAs und OSe gebunden
	\end{itemize}
\end{itemize}


\subsection{Virtualisierung}
\begin{itemize}
	\item Isomorphismus zwischen Gast und Host: Gast-Operationen werden auf enstprechende Host-Operationen überführt und ausgeführt
	\item \textbf{Aufgaben der Virtualisierung}
	\begin{itemize}
		\item Zuordnung von virtuellen zu realen Ressourcen
		\item Verwenden von realen Maschinenbefehlen oder System Calls um die Anweisungen aus dem Gastsystem auszuführen
	\end{itemize}
	\item \textbf{Prozess Virtuelle Maschinen}
	\begin{itemize}
		\item Abstrahiert \textbf{API} oder \textbf{ABI}
		\item Interaktion mit dem Host-Betriebssystem zur Laufzeit via System Calls
		\item Runtime zur Verwaltung der Gastprozesse \(\rightarrow\) Vermischung von Gast- und Hostprozessen
		\item Beispiel: \textit{Wine}
		\item Binary Translation
		\begin{itemize}
			\item Selbes Betriebssystem, verschiedene \texttt{ISAs}: EInführung einer Zwischenschicht zur Übersetzung der \texttt{ABI}-Befehle
			\item Beispiel \texttt{Digital FX!32}: Erlaubt das Ausführen von Windows-Anwendungen, die für \texttt{x86} kompiliert worden sind, auf \texttt{Alpha}-Prozessoren
			\item Beispiel \texttt{HP Dynamo}: Optimierung zur Laufzeit innerhalb der selben Umgebung
		\end{itemize}
		\item Anwendungsbeispiel Java Virtual Maschine: Spezifizierung eines high-level Interfaces. Das Betriebssystem wird als Standard-Bibliothek abstrahiert
	\end{itemize}
	\item \textbf{System Virtuelle Maschinen}
	\begin{itemize}
		\item Abstrahiert die \textbf{ISA} des Hosts als separate virtuelle Maschinen
		\item Interaktion über den \textit{Virtual Machine Monitor} (VMM)
		\item Typen
		\begin{itemize}
			\item Nativ System VM: VMM im privilegierten Modues; Gast im Benutzermodus; beispielsweise \texttt{Xen} oder \texttt{Hyper-V}
			\item Hostet System VM: VMM und Gast im Benutzermodus; beispielsweise \texttt{VirtualBox} oder \texttt{QEMU}
			\item Dual-Mode System VM: VMM teilweise im privilegierten Modus; beispielsweise \texttt{QEMU} mit \texttt{KVM}
		\end{itemize}
		\item Vorteile System VM
		\begin{itemize}
			\item Erhöhte Kompatibilität
			\item Höhere Effizienz, da mehrere virtuelle Maschinen auf einer physischen laufen können
			\item Zuverlässigkeit und Verfügbarkeit wird durch Replikation- und Migrationsmöglichkeiten erhöht
			\item Sicherheit durch Isolierung zwischen den Gästen
			\item Möglichkeiten zur System-Analyse auf \texttt{ISA}-Ebene (Debugging, Malware-Analyse, Forschung)
		\end{itemize}
	\end{itemize}
\end{itemize}